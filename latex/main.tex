\documentclass[a4paper]{article}

\usepackage{url}
\usepackage{fullpage}
\usepackage[percent]{overpic}
\usepackage{graphicx}
\usepackage{framed}
\usepackage{mdframed}
\usepackage{examplep}
\newcommand{\exc}[1]{\mbox{\PVerb{#1}}}
\usepackage{listings}
\usepackage{color}
\definecolor{dkgreen}{rgb}{0,0.6,0}
\definecolor{gray}{rgb}{0.5,0.5,0.5}
\definecolor{mauve}{rgb}{0.58,0,0.82}
\lstset{frame=L,
  aboveskip=3mm,
  belowskip=3mm,
  showstringspaces=false,
  columns=flexible,
  basicstyle={\small\ttfamily},
  numbers=none,
  numberstyle=\tiny\color{gray},
  keywordstyle=\color{blue},
  commentstyle=\color{dkgreen},
  stringstyle=\color{mauve},
  xleftmargin=0.5in,
  breaklines=true,
  breakatwhitespace=true,
  tabsize=3
}
\usepackage[colorlinks=true, urlcolor=blue]{hyperref}

\title{ExamScanUIUC}
\author{Mark C. Bell}

\begin{document}

\maketitle

\begin{center}
\begin{minipage}{0.9\linewidth}
\begin{framed}
To get, install and start the ExamScanUIUC application under Python using Pip:
\begin{lstlisting}
> python -m pip install examscanuiuc --user --upgrade
> python -m examscanuiuc.tag [options] exam.pdf
> python -m examscanuiuc.scan [options] scans.pdf
\end{lstlisting}
\end{framed}
\end{minipage}
\end{center}

ExamScanUIUC is a python package for adding and analysing tags on exams.
It can be run as a Python 2 or Python 3 module.

\section{Getting ExamScanUIUC}

ExamScanUIUC is available on the Python Package Index (PyPI). The preferred method for installing the latest stable release is to use Pip:
\begin{lstlisting}
> python -m pip install examscanuiuc --user --upgrade
\end{lstlisting}
Pip can be installed using \href{http://pip.readthedocs.org/en/latest/installing.html}{\texttt{get-pip.py}} and is included in Python 2.7.9 and Python 3.4 by default.

\subsection{Dependencies}

ExamScanUIUC requires several pieces of software. On Ubuntu these can be installed using:

\begin{tabular}{ll}
\href{https://poppler.freedesktop.org/}{\texttt{pdfimages}} & \texttt{apt-get install poppler-utils} \\
\href{https://poppler.freedesktop.org/}{\texttt{pdftoppm}} &  \texttt{apt-get install poppler-utils} \\
\href{http://zbar.sourceforge.net/}{\texttt{zbar}} & \texttt{apt-get install libzbar-dev libffi-dev}
\end{tabular}

\subsection{ExamScanUIUC development version}

Although the latest stable release of ExamScanUIUC is available through PyPI, you can get the latest development version of flipper from \href{https://bitbucket.org/mark_bell/examscanuiuc}{Bitbucket} or straight from the Mercurial repository with the command:
\begin{lstlisting}
> hg clone https://bitbucket.org/mark_bell/examscanuiuc
\end{lstlisting}
To compile ExamScan use the command:
\begin{lstlisting}
> python setup.py install --user
\end{lstlisting}

\section{Example}

ExamScanUIUC includes an example exam to try tagging and some scans of completed exams.
To copy these into a \texttt{demo} folder in the current directory use the command:

\begin{lstlisting}
> python -m examscanuiuc.demo
\end{lstlisting}

In the following examples, we will assume that you have then moved within this folder by doing:
\begin{lstlisting}
> cd ./demo
\end{lstlisting}

\subsection{Tagging}

To add tags to \texttt{exam.pdf} use the tag module of ExamScanUIUC:
\begin{lstlisting}
	> python -m examscanuiuc.tag exam.pdf
\end{lstlisting}

You will be asked for some basic information, namely the number of exams:
\begin{lstlisting}
	> Number of exams needed: 10
\end{lstlisting}

and the page scores:
\begin{lstlisting}
	> Enter points available per page as a comma separated list (or blank to cancel).
	> Expecting 9 page scores.
	> Unless you want to award points for completing the cover sheet, this should start with a 0: 0,5,6,7,5,5,4,7,6
\end{lstlisting}

ExamScanUIUC will now generate \texttt{output.pdf} containing 10 tagged copies of the exam.

Of course, you can pass this information in directly via flags:
\begin{lstlisting}
	> python -m examscanuiuc.tag exam.pdf --num=10 --scores=0,5,6,7,5,5,4,7,6
\end{lstlisting}

If you want to generate more exams or the exam in several pieces, use the --start flag to set a starting exam number:
\begin{lstlisting}
	> python -m examscanuiuc.tag exam.pdf --num=5 --start=11 --scores=0,5,6,7,5,5,4,7,6
\end{lstlisting}

Alternatively, this information can be specified by a configuration file:
\begin{lstlisting}
	> python -m examscanuiuc.tag exam.pdf --config=settings.conf
\end{lstlisting}

\subsection{Advanced tagging}

This also allows more advanced tagging of the exams by providing a rooms spreadsheet.
\begin{lstlisting}
	> python -m examscanuiuc.tag exam.pdf --config=settings.conf --rooms=rooms.xlsx
\end{lstlisting}
This Excel file contains sheets describing the room layouts and ExamScanUIUC can use these to also add seat names to each exam.

\subsection{Analysis}

To analyse the scanned pages of a tagged exam contained in \texttt{scans.pdf} use the scan module of ExamScanUIUC:
\begin{lstlisting}
	> python -m examscanuiuc.scan --roster=demo/roster.xlsx demo/scan.pdf
\end{lstlisting}

Pages can be scanned in several batches.
Once all pages have been scanned, ExamScanUIUC will write all student data out to \texttt{results.csv}.

Additionally, reports for students can be generated by providing a template:
\begin{lstlisting}
	> python -m examscanuiuc.scam --roster=demo/roster.xlsx --template=demo/template.html
\end{lstlisting}

This html template will have its tags, denoted \texttt{\{\{tag\_name\}\}}, replaced with data for each student.

Temporary data is stored in ./tmp and this should be deleted before starting analysing a new exam.

\section{Known issues}

The packages used by ExamScanUIUC require an updated version of the \href{https://pypi.org/project/six/}{\texttt{six}} package.
Since this is included as an `Extra' package in the included system Python on OS X, Mac users may need to:
\begin{itemize}
\item install Python manually,
\item modify their \texttt{PYTHONPATH} environment variable, or
\item install ExamScanUIUC within \texttt{virtualenv}
\end{itemize}
as described \href{http://stackoverflow.com/questions/29485741/unable-to-upgrade-python-six-package-in-mac-osx-10-10-2}{here}.

\end{document}
